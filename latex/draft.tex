\documentclass{article}
\usepackage[utf8]{inputenc}
\usepackage{amsmath}
\usepackage{graphicx}
\usepackage[style=authoryear, backend=biber]{biblatex} % 使用biblatex
\addbibresource{references.bib} % 确保你的.bib文件名称正确

\title{Your Title Here}
\author{Your Name}
\date{\today}

\begin{document}

\maketitle

\section{Introduction}

  Modern logistics has evolved into a vital enabler of global economic integration, facilitating the seamless flow of goods across production, distribution, and consumption networks. Driven by technological innovation and sustainability imperatives, the industry is undergoing transformative shifts. Recent advancements emphasize the integration of automation and artificial intelligence, with smart warehouses and real-time tracking systems significantly enhancing operational efficiency (\cite{WOS001352081800001}). Concurrently, green logistics initiatives are reshaping traditional practices, prioritizing energy-efficient transportation and recyclable materials to align with global environmental goals (\cite{WOS:000693009900004}).

Drone technology has achieved significant breakthroughs in autonomous navigation and operational capabilities through recent advancements. Modern systems employ multi-sensor fusion architectures combining LiDAR, visual SLAM, and millimeter-wave radar to achieve centimeter-level positioning accuracy in GNSS-denied environments  (\cite{WOS:001140699200001}). The introduction of hydrogen fuel cells have significantly extended operational endurance, overcoming traditional limitations while maintaining eco-friendly profiles  (\cite{WOS:001139514900001}).

Artificial intelligence has revolutionized drone capabilities, empowering intelligent swarm coordination and real-time environmental adaptation. Vision-based navigation techniques enable dynamic route optimization and obstacle avoidance, particularly valuable for tasks like precision agriculture and infrastructure inspection  (\cite{WOS:000939165400001}). Communication breakthroughs, such as 5G-enabled edge computing, provide ultra-low latency control essential for beyond visual line of sight operations and urban air mobility networks  (\cite{WOS:001055864000001}).

Drone technology is reshaping logistics paradigms by introducing aerial solutions to longstanding terrestrial constraints. Unlike conventional delivery methods constrained by road infrastructure, drones leverage low-altitude airspace to establish direct point-to-point connections. This capability proves transformative for last-mile delivery, particularly in geographically challenging regions where traditional transport faces inefficiencies (\cite{WOS:001404731600001}).  By integrating multi-sensor navigation systems and hybrid propulsion technologies, drones enable time-sensitive pharmaceutical deliveries to remote clinics, emergency parts distribution for offshore wind farms, and precision inventory management in smart warehouses (\cite{WOS:001262605900001}). Their ability to bypass terrestrial bottlenecks complements existing logistics networks, creating layered delivery ecosystems where aerial routes dynamically adjust to real-time demand fluctuations. This shift not only addresses critical gaps in rural and urban last-mile logistics but also drives sustainable practices through energy-efficient operations, positioning drones as catalytic enablers for next-generation supply chain resilience.

In the field of drone logistics, literature reviews also play a critical role in synthesizing interdisciplinary advancements, identifying operational gaps, and guiding future research. For instance, Kim et al. (\cite{WOS:001323645000001}) employs text-mining techniques to map two decades of research in drone-assisted multimodal logistics, revealing distinct emphases in drone-truck, drone-ship, and drone-robot systems while highlighting latent topics like energy efficiency and regulatory alignment. Moshref-Javadi and Winkenbach’s framework highlights systemic gaps in multi-drone coordination and resupply dynamics, revealing a disconnect between academic focus on single-drone systems and real-world requirements for complex fleet coordination  (\cite{WOS:000693999200007}). Jazairy et al.’s systematic review bridges logistics management and technical research by evaluating drone potentials across 12 last-mile criteria, such as payload adaptability and noise reduction, while proposing stakeholder-specific solutions like privacy-preserving geofencing for regulators (\cite{WOS:001160422800001}).Jahani et al.’s  review employs text-mining to map drone applications across supply chains, identifying critical gaps in holistic integration studies. Their analysis reveals emerging priorities like sustainable logistics networks and multimodal traffic coordination, while emphasizing the need for pandemic-responsive delivery frameworks (\cite{WOS:001262605900001}).  Law et al.’s review evaluates drone applications in healthcare logistics, emphasizing their transformative potential in crisis response while identifying key barriers such as regulatory fragmentation and public trust  (\cite{WOS:000999662500001}).  

Bibliometrics, a quantitative research method that employs statistical and computational techniques to analyze publication patterns, citation networks, and knowledge evolution within academic domains, serves as a critical tool for mapping interdisciplinary advancements and identifying emerging trends. Its significance lies in synthesizing fragmented research landscapes, revealing collaborative networks, and prioritizing future directions through metrics such as co-citation analysis, keyword clustering, and temporal trend mapping (\cite{WOS:000958189000001}). Within drone technology research, several studies demonstrate the application of bibliometric approaches to map interdisciplinary knowledge domains. For instance, Fayyad et al.   (\cite{WOS:001229751500001}) utilize scientometric analysis to categorize drone-based structural health monitoring (SHM) into four research clusters.  Rejeb et al. (\cite{WOS:000830895100002})) identify remote sensing, precision agriculture, and IoT as dominant themes in agricultural drone studies, while co-citation analysis reveals six research clusters. Iqbal et al.  (\cite{WOS:000914960100001})) highlight evolving priorities in drones for flood management, emphasizing computer vision and real-time detection systems.  Da Silva et al.  (\cite{WOS:001366904300001})) employ bibliometrics to reveal that UAVs are increasingly used over traditional forest inventories to efficiently assess variables like tree height, DBH, biomass, and canopy area through high-resolution sensors. While prior bibliometric reviews like "Analyzing Forklift and Drone Applications in Sustainable Logistics" (\cite{WOS:001329532200078})) provide initial insights into drone logistics, their broad scope—spanning both forklifts and drones—and exclusive focus on sustainability limit granular analysis of drone-specific logistics. In addition, this study remains confined to quantitative metrics, without in-depth review of the literature. It also relies on keyword-based search on data collection, which can be improved by manual screening of the search results. Our study addresses these gaps through a dedicated drone logistics bibliometric analysis, combining quantitative bibliometrics with qualitative analysis of domain-specific innovations. 

The primary objective of this bibliometric study is to systematically map and evaluate the intellectual landscape of drone logistics research, focusing on its evolution, thematic priorities, and collaborative networks. To address this aim, the analysis integrates performance metrics (publication trends, citation impact, leading countries/institutions, prominent journals, and dominant Web of Science categories) with science mapping techniques (keyword co-occurrence, co-authorship networks, bibliographic coupling, and citation dynamics) using data extracted from Scopus and Web of Science. Central research questions include: 


(1) Identification of seminal publications and key contributors shaping the foundation of drone logistics research.
(2) Semantic clustering of research streams and thematic evolution in the field of drone logistics.
(3) Temporal dynamics of collaboration networks and emerging frontiers in drone logistics research.

The paper is structured as follows: The methods section details the data collection protocol, including database selection, search queries, inclusion/exclusion criteria, and analytical tools. The results section presents bibliometric outputs: temporal publication trends, leading contributing countries, high-impact journals, and keyword co-occurrence clusters. The discussions section integrates these metrics with a qualitative synthesis of key literature, categorizing the field into subdomains and evaluating their interconnections, gaps, and evolution over time. The conclusions section summarizes the findings of the study, acknowledges limitations and proposes future research directions.




\section{Methods}
\subsection{Data Collection} % 二级标题
Data extraction was conducted via the Web of Science (WOS) Core Collection. The search strategy combined drone-related terms (TS=("drone*" OR "Unmanned Aerial Vehicle" OR "UAV" OR "Unmanned Aerial System" OR "UAS")) with logistics-focused queries (TS=("logistic*" OR "transport*" OR "supply chain*" OR "delivery" OR "warehouse*")) using the Boolean operator AND, executed on April 1, 2025, yielding 5,649 initial records. Manual screening of titles and abstracts excluded 3,833 publications irrelevant to drone logistics applications (e.g., agricultural monitoring, cinematography). The final dataset comprised 1,816 papers, with metadata fields (authors, institutions, keywords, citations) exported for bibliometric processing.

\section{Results}
This section presents the results

\section{Discussion}
This section discusses the results

\section{Conclusion}
This is the conclusion section

\printbibliography



\end{document}